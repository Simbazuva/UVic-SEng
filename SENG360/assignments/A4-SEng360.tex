%/////////////////////////////////////////////////////////%
%//                        PREAMBLE                     //%
%/////////////////////////////////////////////////////////%
\documentclass[10pt]{article}
%%%%%%%%%%%%%%%%%%%%%%%
%       Packages
%%%%%%%%%%%%%%%%%%%%%%%
%% Fonts and Symbols
%% --------------------------
\usepackage{
    amsmath,            % math operators
    amssymb,            % math symbols
    textcomp,           % Copyright and Registered symbols
    pifont,             % Includes the pretty cirled numbers
    stmaryrd,			% Arrows
}
%% Graphics
%% --------------------
\usepackage{
    graphicx,            % allows insertion of images
    subfigure,           % allows subfigures (a), (b), etc.
    tikz,				% allows drawing shapes and whatnotS
}                
\graphicspath{ {graphics/} }    % (graphicx) relative path to graphics folder    
%% Tables
%% --------------------------
\usepackage{
    booktabs,            % better tables, discourages vertical rulings
    multicol,            % allow multi columns
    tabularx,            % variable-width columns
}
%% Layout Alteration
%% --------------------------
\usepackage{            
    geometry,            % change the margins for specific PAGES
    parskip,             % no indenting on paras with a line between paras
    fancyhdr,            % fancy headers and footers
}
\geometry{               % specify page size options for (geometry)
    letterpaper,         % paper size
    margin=1in,			% specified independently with hmargin vmargin
}    
%% Units
%% --------------------------
\usepackage{
    siunitx,             % has S (decimal align) column type
}
\sisetup{input-symbols = {()},  % do not treat "(" and ")" in any special way
    group-digits  = false, % no grouping of digits
    %         load-configurations = abbreviations,
    %         per-mode = symbol,
}
%% Misc
%% --------------------------
\usepackage{
    pdfpages,            % import pdfs into this document
    url,                % allows urls to be added to document
    amssymb,
    amsmath,
    listings,
    tikz,
}
%/////////////////////////////////////////////////////////%
%//                        BODY                         //%
%/////////////////////////////////////////////////////////%

\begin{document}
	%%%%%%%%%%%%%%%%%%%%%%%
	%       Title Section
	%%%%%%%%%%%%%%%%%%%%%%%
	\pagenumbering{gobble}        % turn off page numbering  
	\begin{center}
		\begin{tabularx}{\textwidth}{>{\raggedright}X>{\setlength\hsize{1\hsize}\centering}X>{\raggedleft}X}     
			SEng360            &    {\huge Assignment 4 }            &    Jakob Roberts\tabularnewline
			&    {\small  }              		  &    v00484900\tabularnewline
		\end{tabularx}    
	\end{center}  
	%%%%%%%%%%%%%%%%%%%%%%%
	% 	  Body
	%%%%%%%%%%%%%%%%%%%%%%%
	\vspace{10mm}
	\renewcommand{\arraystretch}{1.6}
	\section*{A}
	\begin{itemize}
		\item []
		\usetikzlibrary{positioning}
		\begin{tikzpicture}[node distance=3cm]
		\node(L1AO)                          {$$};
		\node(L1A)        [below right = 1cm and 7.5cm of L1AO]    {$L_1\alpha$};
		\node(L1O)        [below right = 1cm and 7.5cm of L1A]      {$$};
		
		\node(L2AO)       [below of=L1AO] {$L_2\alpha\omega$};
		\node(L2A)      [below of=L1A]  {$$};
		\node(L2O)      [below of=L1O]  {$L_2\omega$};
		
		\node(L3AO)       [below of=L2AO] {$L_3\alpha\omega$};
		\node(L3A)      [below of=L2A]  {$$};
		\node(L3O)      [below of=L2O]  {$L_3\omega$};		
		
		\node(L4AO)       [below of=L3AO] {$$};
		\node(L4A)      [below of=L3A]  {$L_4\alpha$};
		\node(L4O)      [below of=L3O]  {$L_4\omega$};		
		
		\node(L5AO)       [below of=L4AO] {$$};
		\node(L5A)      [below of=L4A]  {$L_5\alpha$};
		\node(L5O)      [below of=L4O]  {$L_5\omega$};			
		
		\draw[-](L2AO)		--	(L3AO);
		\draw[-](L1A)       -- (L4A);
		\draw[-](L4A)		--	(L5A);
		\draw[-](L2O)		--	(L3O);
		\draw[-](L3O)		--	(L4O);
		\draw[-](L4O)		--	(L5O);
		
		\draw[-](L2AO)		--	(L2O);
		\draw[-](L3AO)		--	(L3O);
		\draw[-](L3AO)		--	(L4A);
		
		\end{tikzpicture}
	\end{itemize}
	%%%%%%%%%%%%%%%%%%%%%%%
	%\pagebreak
	%%%%%%%%%%%%%%%%%%%%%%%
	\section*{B}
	\begin{itemize}
		\item []
		\begin{tabular}{ | l | l | l |}
			\hline
			\multicolumn{1}{|c|}{\textbf{Subject}} & \multicolumn{1}{|c|}{\textbf{Object}} & \multicolumn{1}{c|}{\textbf{Access}} \\ \hline
			Alice & P2 & None \\ \hline
			Bob & P2 & Read/Write \\ \hline
			Ellen & P4 & Write \\ \hline
			Dave & P5 & Read \\ 
			\hline
		\end{tabular}
	\end{itemize}
	%%%%%%%%%%%%%%%%%%%%%%%
	%\pagebreak
	%%%%%%%%%%%%%%%%%%%%%%%
	\section*{C}
	\begin{itemize}
		\item[]
		\begin{tabular}{| l || l | l | l |}
			\hline
			& \multicolumn{1}{c|}{\textbf{Obj1}} & \multicolumn{1}{c|}{\textbf{Obj2}} & \multicolumn{1}{c|}{\textbf{Obj3}} \\ \hline\hline
			User1 & R & \(\varnothing\) & R \\ \hline
			User2 & \(\varnothing\) & R & R \\ \hline
			User3 & R & R & R \\
			\hline
		\end{tabular}
	\end{itemize}
\end{document}
